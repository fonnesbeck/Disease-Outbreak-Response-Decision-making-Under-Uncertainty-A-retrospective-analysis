\textbf{Figure 3} A) The estimated total number of susceptible
individuals in each age class based on data observed on 15 July using
the age confirmation model (orange) and the clinical only model (green).
B) The estimated number of resident susceptibles - total susceptibles
minus excess susceptibles - in each age class based on data observed on
15 July using the age confirmation model (orange) and the clinical only
model (green).

The effective reproduction number, which describes the rate at which the
epidemic spread in 1997 in a partially immune population was estimated
to be significantly greater than 1 for on both 15 June and 15 July using
the age confirmation model (Figure 4). Notably, while the posterior mean
estimate for RE on 15 June was 1.125, the 95\% credible did include the
estimate including the data through 15 July. Ignoring the age
confirmation model and treating all clinical cases as measles positives
yielded a 15 June estimate of RE = 1.02 with a 95\% credible interval
(0.95-1.08) that includes 1. Using all data through 15 July, the
estimate of the RE was comparable whether clinical cases were corrected
using the age confirmation model or not.