\section{Introduction}\label{introduction}

Reports of resurgent outbreaks of vaccine-preventable diseases following long periods of relative absence are  increasingly common \cite{Hersh_1991, Cherry_2012, Celentano_2005, Shibeshi_2014}. Several factors may contribute to the occurrence of such outbreaks. McLean and Anderson \cite{McLean_1988} predicted that such outbreaks should be expected because of the ``honeymoon'' phenomenon following the introduction of vaccination, whereby post-vaccination cohorts no longer experience high rates of natural immunization to supplement population immunity following vaccination activities \cite{Jansen_2003}. Further, population-level vaccination rates may decline over time as immigration from areas of low vaccination coverage lead to a buildup of susceptible individuals, or the reduction in individual infection risk \cite{Omer_2009} leads to apathy about vaccination within the population. Also, local stochastic extinction may result in temporary breakdown of local transmission, even though populations remain susceptible to subsequent outbreaks upon the reintroduction of infection \cite{Ferrari_2008}.

Increasingly, in the event of a measles outbreak, outbreak response immunization (ORI) is recommended as an intervention. The goals of these ORIs are two-fold: 1) to protect high risk groups (\emph{i.e.} young children) and 2) to attenuate the current outbreak \cite{Cairns_2011,Grais_2011}. To achieve the former goal, ORI campaigns routinely target children 6-59 months of age \cite{Cairns_2011}. To achieve the latter goal, the campaign must reach some target level of immunization (\(P_c\)) --- \emph{i.e.} a percentage reduction of the susceptible population --- such that effective reproductive number, \(R_e\), will be below 1 and the outbreak will end. From the standard susceptible-infected-removed (SIR) model, this level of immunization is \(P_c = 1-1/R_e\) \cite{Anderson_1981}. To identify this target coverage and appropriately plan a campaign, one must estimate both the value of \(R_e\) itself, which determines the necessary reduction of the susceptible population required to end the outbreak, and the age distribution of the susceptible population, which allows us to identify the critical age classes to be targeted in a campaign. For example, if \(R_e=1.5\) one must then reduce the susceptible fraction by 33\%; if 80\% immunization of susceptibles in the target age groups can be achieved, then at a minimum, the intervention should target individuals up to the 100*(0.33/0.8) = 41.25th percentile of the susceptible population age distribution.

Experience with past outbreaks can provide guidance about likely values of \(R_{e}\) \cite{Durrheim_2014} and the likely distribution of the susceptible population \cite{Goodson_2011, 25803382}. However, in the case of resurgent outbreaks, which follow periods of relatively low measles incidence, there may be insufficient data on which to base estimates of \(R_e\). In these settings, it is the current outbreak itself which may provide the most relevant empirical information \cite{Durrheim_2014, Merl_2009, Shea_2014}. However, clinical confirmation of measles cases through case-based surveillance systems has relatively low specificity, particularly in settings of low prevalence \cite{Hutchins_2004,Ho_2014,GUY_2004,31c964} when other illnesses that result in fever and rash (e.g. rubella, Dengue) may be misdiagnosed as measles \cite{Ho_2014}. Given that the various etiologies of febrile illness may disproportionately affect different age classes, reliance on a clinical definition alone may result in a biased assessment of the age classes at risk \cite{Hutchins_2004,Durrheim_2014}. Further, misdiagnosis of cases may bias the assessment of the rate of increase of total cases and lead to a biased estimate of \(R_{e}\) and hence the vaccination coverage necessary to limit the outbreak. Though serological confirmation of measles cases is preferred to clinical confirmation, resources often limit the proportion of clinical cases that can be confirmed by serology in outbreak settings in time to be of use to decision-makers. 

A variety of methods are available to estimate \(R_{e}\) from early surveillance data \cite{Durrheim_2014,Chiew_2013,Ferrari_2005,Lipsitch_2003}. The use of mathematical modeling to estimate the age distribution of the susceptible population during an outbreak is rare; decisions about age targeting have been classically based on prior experience or early evaluation of cases confirmed earlier in the outbreak \cite{Minetti_2013a}. Here we present an epidemic model that uses serological confirmation on a subset of cases to estimate age-specific confirmation, and use these estimates to correct the observed number of reported cases before being used to estimate epidemic parameters.

The age distribution of the susceptible population is usually unknown at the start of a measles outbreak. The mean and variance of the age distribution of susceptible individuals is expected to increase as the prevalence of infection declines \cite{Goodson_2011, Ferrari_2013} and during periods of measles absence \cite{Durrheim_2014}. As a result, historical surveillance data may be misleading with respect to the current distribution of susceptibles. Initial estimates of the susceptible population can be reconstructed from demographic rates, historical records of routine and supplemental vaccination coverage, and measles incidence\cite{Takahashi_2015}; however, uncertainty in these rates and historical incidence mean that \emph{a priori} estimates of the susceptible population may be significantly biased. Further, lack of data on migration, heterogeneity of vaccination coverage, and clustering of susceptibles means that these estimates will be highly uncertain in the best of circumstances. The 1997 measles outbreak in Sao Paulo, Brazil, presented in detail below had many more adolescent and adult cases than was expected based on historical rates \cite{Camargo_2000}. This pattern of unexpectedly wide age distributions of cases has been recently seen in outbreaks in Malawi \cite{Minetti_2013} and Mongolia (ref: http://www.wpro.who.int/mongolia/mediacentre/releases/20160505-measles-outbreak-faqs/en/).

This work comprises a retrospective analysis of a resurgent measles outbreak in Sao Paulo, Brazil in 1996-1997. This outbreak followed several years (give years) of relatively low measles incidence as a result of both routine and supplemental measles vaccination. The outbreak itself resulted in over 30,000 confirmed cases in Sao Paulo State, with an unexpectedly high proportion of cases in adults; 60\% were in individuals greater than 20
years of age. Several limited vaccination campaigns, targeting children under 4 years of age, health workers, and some adults were implemented between June and August, prior to a widespread campaign targeting all children between 6 months and 4 years of age that was conducted in August of 1997. As a consequence of the broad age distribution of susceptibles, typical childhood-based ORIs may not have resulted in sufficient immunity to limit the outbreak. We created a novel statistical model that combines an \emph{a priori} model of the susceptible population based on available immunization coverage information, a time series model of the progression of the outbreak, and an age-specific model of IgM serological confirmation of suspected cases to estimate \(R_{e}\)and the age distribution of the susceptible population. To illustrate how real-time surveillance could be applied to inform the design of ORI age targets, we generated estimates of \(R_{e}\) and the resulting ORI target recommendations at both 1 and 2 months prior to the vaccination campaign that was conducted on 15 August. We argue that a flexible approach to ORIs can better incorporate the information gained in the early stages of an
outbreak to identify campaign age targets and that unforeseen biases in clinical diagnosis can be mitigated through the incorporation of serological confirmation and high quality surveillance data.