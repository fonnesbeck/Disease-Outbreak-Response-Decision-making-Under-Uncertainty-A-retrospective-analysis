\section{Introduction}\label{introduction}

Resurgent outbreaks of childhood infections following long periods of
relative absence are an increasingly common phenomenon \cite{Hersh_1991, Cherry_2012, Celentano_2005, Shibeshi_2014}. Several factors may contribute to the occurrence of
such outbreaks. McLean and Anderson \cite{McLean_1988} predicted that such
outbreaks should be expected because of the ``honeymoon'' phenomenon
following the introduction of vaccination whereby post-vaccination
cohorts no longer experience high rates of natural immunization to
supplement vaccination \cite{Jansen_2003}. Further, population-level vaccination rates may
decline over time as the reduction in individual infection risk (REF)
leads to apathy within the population, or as immigration from areas of
low immunity becomes substantial (REF). In addition, local stochastic
extinction may result in temporary breakdown of local transmission, even
though populations remain susceptible to subsequent outbreaks upon the
reintroduction of infection \cite{Ferrari_2008}.

Increasingly, in the event of a measles outbreak, outbreak response
immunization (ORI) has been recommended as an intervention. The goals of
these ORIs are two-fold: 1) to protect high risk groups (\emph{i.e.}
young children) and 2) to attenuate the current outbreak \cite{Cairns_2011,Grais_2011} . To achieve the former goal, ORI campaigns routinely
target children between the ages of 9-59 months of age\footnote{though
  there has been increasing discussion about vaccinating children from
  6-9m}. To achieve the latter goal, the campaign must reach some
target level of immunization --- \emph{i.e.} a percentage reduction of
the susceptible population --- such that effective reproductive number,
\(R_e\), will be below 1 and the outbreak will end. From the standard SIR
model, this level of immunization is \(P_c = 1-1/R_e\) \cite{Anderson_1981}. To identify this target coverage and appropriately plan a campaign, one must estimate
both the value of \(R_e\) itself and the age distribution of the
susceptible population, which determines the necessary reduction of
the susceptible population required to end the outbreak and the
critical age classes to be targeted in a campaign, respectively.
For example, if \(R_e=1.5\) one must then reduce the
susceptible fraction by 33\%; if one can reasonably assume to achieve
80\% immunization of susceptibles in the target age groups then individuals up to the age that is the the
100*(0.33/0.8) = 41.25th percentile of the susceptible population should, at minimum, be targeted.

Experience with past outbreaks can provide guidance about likely values
of \(R_{e}\) \cite{Durrheim_2014} and the likely distribution of the
susceptible population \cite{Goodson_2011, 25803382}. However, in
the case of resurgent outbreaks, which follow periods of relatively low
measles incidence, there may be insufficient data on which to base
estimates. In these settings, it is the current outbreak itself which
may provide the most relevant empirical data \cite{Durrheim_2014, Merl_2009, Shea_2014}. A variety of methods are available to estimate \(R_{e}\) from early surveillance data \cite{Durrheim_2014} (REF Durheim 2014 {[}references
within -\/- Gay et al 2004, Chiew et al 2014, Lim et al 2014{]}, Ferrari
et al 2006, Lipsitch et al 2003). Formal evaluation of the age
distribution of the susceptible population during an outbreak has been
less common; decisions about age targeting have been classically based
on prior experience or early evaluation of clinically confirmed cases \cite{Minetti_2013a}.

The age distribution of the susceptible population is rarely known at
the start of a measles outbreak. The mean and variance of the age
distribution of susceptible individuals is expected to increase as the
prevalence of infection declines \cite{Goodson_2011, Ferrari_2013} and
during periods of measles absence \cite{Durrheim_2014}. As a result,
historical surveillance data may be misleading as to the current
distribution of susceptibles. Initial estimates of the susceptible
population can be reconstructed from demographic rates, historical
records of routine and supplemental vaccination coverage, and measles
incidence; however, uncertainty in these rates and historical incidence
mean that \emph{a priori} estimates of the susceptible population may be
significantly biased. The 1997 measles outbreak in Sao Paulo, Brazil,
presented in detail below had many more adolescent and adult cases than
was expected based on historical rates {[}REF{]}. This pattern of
unexpectedly wide age distributions of cases has been recently seen in
outbreaks in Malawi {[}REF: Minetti{]} and Mongolia {[}REF{]}.

Clinical confirmation of measles cases has relatively low specificity,
particularly in settings of low prevalence (REF Deitz et al, Guy et al,
Hutchins et al, Ho et al ) when other illnesses that result in fever and
rash (e.g. rubella) may be misdiagnosed as measles (Ho et al 2014,
Mancini 2014). Given that the various sources of febrile illness may
disproportionately affect different age classes, reliance on a clinical
definition alone may result in a biased assessment of the age classes at
risk (REF Hutchinson et al, Durrheim 2014). Further, misdiagnosis of
cases may bias the assessment of the rate of increase of total cases and
lead to a biased estimate of \(R_{e}\) and hence the vaccination
coverage necessary to limit the outbreak. Though serological
confirmation of measles cases is preferred to clinical confirmation,
resources often limit the proportion of clinical cases that can be
confirmed by serology in outbreak settings. Here we present an epidemic
model that uses serological confirmation on a subset of cases to

We present a retrospective analysis of a resurgent measles outbreak in
Sao Paulo, Brazil in 1996-7. This outbreak followed several years of
relatively low measles incidence as a result of both routine and
supplemental measles vaccination. The outbreak itself resulted in over
30,000 confirmed cases in Sao Paulo State, with an unexpectedly high
proportion of cases in adults; 60\% were in individuals greater than 20
years of age. As a consequence, typical childhood-based ORIs would not
be expected to result in sufficient immunity to limit the outbreak. We
develop a novel statistical model that combines an \emph{a priori} model
of the susceptible population, a time series model of the progression of
the outbreak, and an age-specific model of IgM serological confirmation
of suspected cases to estimate \(R_{e}\)and the age distribution of the
susceptible population. To illustrate how real-time surveillance could
be applied to inform the design of ORI age targets, we show generate
estimates and the resulting ORI target recommendations at time points 1
and 2 months prior to the non-selective vaccination campaign that was
conducted on 15 August. We argue that a flexible approach to ORIs can
better incorporate the information gained in the early stages of an
outbreak to identify campaign age targets and that unforeseen biases in
clinical diagnosis can be mitigated through the incorporation of
serological confirmation.