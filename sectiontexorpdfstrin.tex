\subsection{\texorpdfstring{\emph{Confirmation Bias
Model}}{Confirmation Bias Model}}\label{confirmation-bias-model}

We specified a structured case confirmation submodel to retrospectively
determine the age group-specific probabilities of lab confirmation (\emph{i.e.}
lab positive for measles specific IgM) for measles in Sao Paulo,
conditional on clinical diagnosis. Individual lab confirmation events
\(c_i\) were modeled as Bernoulli random variables,
with the probability of confirmation being allowed to vary by age group:

\begin{equation}
c_{i} \sim Bernoulli(p_{a[i]})\]
\end{equation}

where \(a[i]\) denotes the age group for the individual
indexed by \emph{i}. There were 16 age groups, the first 15 of which
were 5-year age intervals \emph{{[}0,5), {[}5, 10), \ldots{} , {[}70,
75)}, with the last interval including all individuals 75 years and
older. Since our choices of age group boundaries were arbitrary, we
allowed the probabilities of adjacent groups to be correlated with one
another. To this end, the transformed set of probabilities was
modeled as a multivariate normal random variable: 

\begin{equation}
\beta^{(conf)} \sim N(\mu, \Sigma)
\end{equation}

where the transformation is the logit (\(\text{logit}(p) = \log[p / (1-p)] \)), which serves to convert probabilities (defined on the [0,1] interval) to the real line:

\begin{equation}
\text{logit}(p_a) = \beta_a^{(conf)}
\end{equation}

where the corresponding covariance matrix was tridiagonal, incorporating a
correlation term \(\rho\) that was assumed, for simplicity, constant among groups. This was to allow confirmation rates to be correlated between neighboring age groups, since the age boundaries were arbitrarily defined.

\begin{equation}
\Sigma = \left[{
\begin{array}{cccccc}
  {\sigma^2} & {\sigma^2 \rho} & 0& \ldots & {0} & {0}  \\
  {\sigma^2 \rho} & {\sigma^2} &  \sigma^2 \rho & \ldots & {0}  & {0} \\
  {0} & \sigma^2 \rho & {\sigma^2} & \ldots & {0} & {0} \\
  \vdots & \vdots & \vdots &  & \vdots & \vdots\\
  {0} & {0} & 0 & \ldots &  {\sigma^2} & \sigma^2 \rho  \\
{0} & {0} & 0 & \ldots & \sigma^2 \rho &  {\sigma^2} 
\end{array}
}\right]
\end{equation}


To estimate the true (latent) number of cases for each age group \(I_a\), the estimated probabilities were used to correct the clinically-reported cases \(n_a^{(clinic)}\), as modeled by a binomial distribution, and the total number of cases then calculated as the sum of this estimated value \(x_a^{(clinic)}\) and the lab-confirmed cases \(x_a^{(lab)}\):

\begin{eqnarray}
x_a^{(clinic)} &\sim& \text{Bin}(n_a^{(clinic)},p_a) \\
I_a &=& x_a^{(clinic)}+x_a^{(lab)}
\end{eqnarray}

The set of \(I = \{I_a\}\) was, in turn, used to inform the estimate of the latent age distribution of the infected class. A natural probability distribution to model age classes is the multinomial distribution, which is parameterized by a set of probabilities, here corresponding to the expected proportion in each age group; the corresponding prior for this vector of probabilities is the  Dirichlet distribution. These are specified by:

\begin{equation}
\begin{aligned}
f^{(age)} &\sim \text{Dirichlet}(\mathbf{1}) \\
\{I_a\} &\sim \text{Multinomial}(I, f^{(age)})
\end{aligned}
\end{equation}

where \(f^{(age)}\)is a vector of age-specific proportions and \(\mathbf{1}\) a vector of ones, which is used as a non-informative prior for the Dirichlet.