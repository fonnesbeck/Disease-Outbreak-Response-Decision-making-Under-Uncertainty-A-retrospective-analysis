\section{\texorpdfstring{\textbf{Methods}}{Methods}}\label{methods}

\subsection{\texorpdfstring{\emph{Data}}{Data}}\label{data}

Case-based records of individuals presenting with clinical measles
symptoms (fever, rash, and at least one of the following symptoms cough,
conjunctivitis, or coryza REF) during the calendar year of 1996 were
provided by the Ministry of Health (CORRECT TITLE). Fields for each
record included the county in which the case presented, date of
notification to the health facility, age in months or years, and the
results (positive, negative, or inconclusive) of an serological test for
measles specific IgM if one was conducted. After subsetting to cases
presenting in urban Sao Paulo and discarding those with incomplete
records, there were 10,810 cases presenting with clinical symptoms only
and 23,699 cases with serological tests, of which 1067 were discarded
due to lack of reagent or improper collection and thus treated as
clinical cases.

The age distribution of the population in Sao Paulo was extrapolated
from the decadal census. Historical rates of routine vaccination
coverage were taken from (give WHO web address for historical coverage).
The coverage of supplemental immunization activities (SIAs) conducted in
1987 and 1992 were assumed to be 90\%.

\subsection{\texorpdfstring{\emph{Confirmation Bias
Model}}{Confirmation Bias Model}}\label{confirmation-bias-model}

We specified a structured case confirmation submodel to retrospectively
determine the age group-specific probabilities of lab confirmation, i.e.
lab positive for measles specific IgM, for measles in Sao Paulo,
conditional on clinical diagnosis. Individual lab confirmation events
\emph{c\textsubscript{i}} were modeled as Bernoulli random variables,
with the probability of confirmation being allowed to vary by age group:

\[c_{i} \sim Bernoulli(p_{a\lbrack i\rbrack})\]

where \(a\lbrack i\rbrack\) denotes the age group for the individual
indexed by \emph{i}. There were 16 age groups, the first 15 of which
were 5-year age intervals \emph{{[}0,5), {[}5, 10), \ldots{} , {[}70,
75)}, with the 16th interval including all individuals 75 years and
older. Since our choices of age group boundaries were arbitrary, we
allowed the probabilities of adjacent groups to be correlated with one
another. To this end, the logit-transformed set of probabilities was
modeled as a multivariate normal random variable, where the
corresponding covariance matrix was tridiagonal, incorporating a
correlation term \(\rho\)that was assumed constant among groups.

\(
\Sigma = \left[{
\begin{array}{c}
  {\sigma^2} & {\sigma^2 \rho} & 0& \ldots & {0} & {0}  \\
  {\sigma^2 \rho} & {\sigma^2} &  \sigma^2 \rho & \ldots & {0}  & {0} \\
  {0} & \sigma^2 \rho & {\sigma^2} & \ldots & {0} & {0} \\
  \vdots & \vdots & \vdots &  & \vdots & \vdots\\
  {0} & {0} & 0 & \ldots &  {\sigma^2} & \sigma^2 \rho  \\
{0} & {0} & 0 & \ldots & \sigma^2 \rho &  {\sigma^2} 
\end{array}
}\right]\)

\begin{eqnarray}
\beta^{(conf)} &\sim& N(\mu, \Sigma) \\
\text{logit}(p_a) &=& \beta_a^{(conf)}
\end{eqnarray}

To estimate the true number of cases for each age group, the estimated probabilities were used to correct the clinic-reported cases, as modeled by a binomial distribution, and the total number of cases then calculated as the sum of this corrected value and the lab-confirmed cases:

\begin{eqnarray}
\text{df}x_a^{(clinic)} &\sim& \text{Bin}(n_a^{(clinic)},p_a) \\
I_a &=& x_a^{(clinic)}+x_a^{(lab)}
\end{eqnarray}

This was used to inform the estimate of the latent age distribution of susceptibles, modeled as a Dirichlet distribution:

\begin{eqnarray}
f^{(age)} &\sim& \text{Dirichlet}(\mathbf{1}) \\
I &\sim& \text{Multinomial}(f^{(age)})
\end{eqnarray}

where \(f^{(age)}\)is a vector of age-specific proportions and \(\mathbf{1}\) a vector of ones, used as a non-informative prior for the Dirichlet.