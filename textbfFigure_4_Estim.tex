These differences in $R_e$, with posterior means ranging from 1.02 - 1.25 depending on time (June or July) and observation model (clinical or age-corrected), though small in absolute value, reflect larger practical differences in the vaccination response required to stop the outbreak. If we use these estimates to calculate minimum vaccination thresholds (assuming campaigns that reach 90\% of the target population and achieves 95\% effectiveness), this establishes vaccination targets ranging from 2 to 20\% of susceptibles; conservatively using the 95th percentile of the posterior distribution of $R_e$ estimates, this shifts the target range to
8 - 24\% (Figure 5). From this, we define an ORI campaign as \emph{sufficient} if it is expected to reduce the susceptible population by at least the threshold amount. Based on the July estimates, neither confirmation model predicts a 5-and-under strategy to be sufficient for stopping the outbreak (Figure 5). Using the June estimates under the age confirmation model, there is the suggestion that targeting children under 5 years would be
sufficient, while the June estimate fit to the clinical cases only predicts all strategies would be comfortably above the threshold. In all cases, ORI strategies that target individuals up to 15 or 30 years of age, as well as a mixed strategy that targets children under 5 and adults between 20-30 years of age, are predicted to be sufficient to stop the outbreak.