Resurgent outbreaks of vaccine-preventable diseases that have previously been controlled or eliminated have been observed in many settings. Reactive vaccination campaigns may successfully control outbreaks but must necessarily be implemented in the face of considerable uncertainty.  Real-time surveillance may provide critical information about at-risk population and optimal vaccination targets, but may itself be limited by the specificity of disease confirmation. We propose an integrated modeling approach that synthesizes historical demographic and vaccination data with real-time outbreak surveillance via a dynamic transmission model and an age-specific disease confirmation model. We apply this framework to data from the 1996-7 measles outbreak in São Paulo, Brazil. To simulate the information available to decision-makers, we truncated the surveillance data to what would have been available at 1 or 2 months prior to the realized interventions. We use the model, fitted to real-time observations, to evaluate the likelihood that candidate age-targeted interventions could control the outbreak. Using only data available prior to the interventions, we estimate that a significant excess of susceptible adults would prevent child-targeted campaigns from controlling the outbreak and that failing to account for age-specific confirmation rates would under-estimate the importance of adult targeted vaccination.