\section{Discussion}\label{discussion}

The 1997 measles outbreak in Sao Paulo was unexpected, having followed several years of high routine vaccination coverage, SIAs, and relatively low
incidence. Further, the age distribution of cases, with a secondary mode among adults, had not been observed in previous outbreaks. Thus, while
historical precedent often serves as a guide for outbreak response, in this case, historical precedent would have greatly under-estimated
outbreak risk and the vaccination targets necessary to limit the outbreak. Here we have presented a novel approach for integrating
real-time outbreak surveillance into the evaluation of an evolving outbreak in order to evaluate candidate response strategies. In doing
so, we have developed a novel model for interpreting clinical measles surveillance that acknowledges that the correlation between clinical
measles symptoms and lab confirmation of positive measles IgM serology is age-specific. Further, we have shown that relying on clinical
confirmation alone can significantly bias inference about transmission rate ($R_E$) and the minimal vaccination targets to stop an outbreak.

In the case of Sao Paulo in 1997, estimates of $R_E$ and the likelihood that different age-targeted vaccination campaigns would meet the
necessary immunization threshold were similar (\emph{e.g.} under-5y campaigns were estimated to be insufficient to limit the outbreak), regardless of model used, when fit to data on 15 July. Thus, while our proposed model accounting for age-specific bias in serologic confirmation explicitly estimates the uncertainty in clinical diagnosis, it results in no practical difference in the interpretation of risk \(R_E\) or candidate interventions on 15 July. However, using only the data available on 15 June, estimates based only on clinical confirmation data would have grossly under-estimated risk and over-estimated the benefit of a vaccination campaign targeting children below 5 years of age. Estimates based on our proposed model, which incorporate age-specific confirmation rates, result in wider interval estimates of $R_E$ that include the
estimates based on the 15 July data. Thus, while less precise (i.e.
wider interval), the model with age-specific confirmation does not
exclude the transmission rate that would would have been made with an
additional 1 month of observed data. This is of practical relevance,
because it means that the estimated impact of a vaccination campaign
targeting only children under 5 years would have overlapped the
vaccination coverage necessary, i.e. would have been insufficient, to to
limit the outbreak to \(R_E \lt 1\). Notably the data available on 15
July clearly indicated that a campaign targeting only children under 5
years would have been insufficient.

Though outbreak risk can be evaluated \emph{a priori}, outbreaks themselves are
often the first indication of the build-up of susceptibles or gaps in
immunity. In Sao Paulo in 1997, the age distribution of cases indicated
a significant gap in immunity among individuals between 15-35 years of
age. The SIA conducted in 1987, targeting children below 14 years of
age, would be expected to have immunized individuals below 23 years of
age, and those older than 23 years would have been born prior to a
national immunization system in Brazil {[}REF{]} and would be expected
to have experienced natural infection during their childhood. We found
that excess susceptibles between 15-35 years of age accounted for 60\%
{[}verify exact value and CI{]} of all susceptibles during the 1997
outbreak. We term these as excess susceptibles because they are
inconsistent with the expected age distribution of susceptibles based on
historical rates of natural infection, routine vaccination, and SIAs. It
is impossible to identify an exact source for these excess susceptibles;
they may be the result of over-estimates of the coverage of previous
vaccination programs, or immigrants from low coverage and low
transmission risk areas that were unlikely to be exposed to vaccination
or natural infection. While the former explanation is possible, lower
than nominal vaccination coverage would be expected to result in more
circulating infection, which would still likely result exposure to
natural infection, and thus immunity, by adulthood. The latter
explanation requires that individuals were recent immigrants to Sao
Paulo and had not been exposed to either vaccination or natural
infection as children in the region that the emigrated from. Measles
rarely persists endemically in small populations below some critical
community size {[}REF{]}; thus it is possible that recent immigrants
from small villages might have not been exposed to natural infection.
Camargo et al {[}REF{]} conducted a case-control trial after the 1997
outbreak and found that recent immigration to Sao Paulo was a
significant risk factor for measles infection during the outbreak.
Further, immigration rates into Sao Paulo in 1991 were highest among
individuals between 15 and 30 years of age {[}REF de Moraes and Fundacao
Seade{]}, which is consistent with the age distribution of the excess
susceptibles from our fitted model. While this does not confirm that
immigration or gaps in prior immunization were the source of the adult
susceptibles during the 1997 outbreak, this analysis does indicate that
these adult susceptibles played a significant role in the outbreak;
removing the excess susceptibles would have resulted in a reduction of
R\_E at the start of the outbreak to less than 1. Other recent measles
outbreaks have exhibited this same age-profile, with an unexpectedly
large number of adult cases; e.g. Malawi \cite{Minetti_2013}, Mongolia
{[}REF{]}, China {[}is there a REF?{]}. Thus, strategies for monitoring
and targeting immunity gaps in adults may be useful in preventing future
outbreaks and outbreak response strategies should consider
adult-targeted vaccination when surveillance indicates a large number of
adult susceptibles.

Though our models consider the age distribution of susceptibles we make
very simplistic assumptions about age-specific transmission; namely that
within age-class transmission is the same for all ages, and between
age-class transmission decays exponentially with difference in ages.
Considerable recent work has shown that age-specific mixing rates are
likely to vary considerably and may be culturally specific  {[}REF
Massong, etc{]}. It is possible that higher contact, and thus
transmission, rates among adults means that adult susceptibles
disproportionately contributed to this outbreak. Fitting an age-specific
transmission matrix with differential age-specific rates would have
greatly inflated the number of unknown parameters and is beyond the
scope of this work. {[}reference to Samit's paper if its out{]}

There are many potential explanations for the age-specific serological
confirmation bias for cases with clinical symptoms. There are many
etiologies that may generate fever and rash symptoms in young children
{[}REF -\/- Orenstein?{]}, which would increase the rate of false
positives in children based on clinical symptoms alone. Further, young
children may be brought to clinic shortly after the onset of symptoms,
when IgM titers may not yet have reached detectable levels. Regardless
of the cause of this bias, the result is that the ability to assess the
overall age distribution of cases may tend to underestimate the role of
adults if based on clinical confirmation alone. Further, as the age
distribution of cases presenting with symptoms may change over the
course of an outbreak (give examples from this outbreak), clinical cases
may reflect time-varying confirmation and bias estimates of transmission
rates. Here, by correcting for age-specific confirmation bias, we also
necessarily correct for temporal variation in false positives. We see
the effect of this in the higher estimated \(R_E\) using the confirmation
bias corrected model with the 15 June data; though the model predicts
fewer true measles cases (\emph{i.e}. false positives are removed), the
exponential increase from the start of the outbreak to 15 June is
greater and this estimate is consistent with the higher \(R_E\) estimate
based on the longer time series available on 15 July.

Something about incorporating serology in to outbreak surveillance and
real-time evaluation

Main points - to be fleshed out upon finalization of above.

\begin{enumerate}
\def\labelenumi{\arabic{enumi}.}
\item
  \begin{quote}
  Outbreak was unexpected -\/- interventions based on expected age
  distribution of susceptibles would have been insufficient, didn't
  expect adult susceptibles.
  \end{quote}
\item
  \begin{quote}
  Observation of outbreak depends on both reporting rate AND
  confirmation rate. The latter is age-dependent. Given that age
  distribution of cases was time dependent, then observation rate is
  also time dependent. We present a model that can formally account for
  time dependence in the observation rate due to age-specific
  confirmation.
  \end{quote}
\item
  \begin{quote}
  The estimates of RE and the age distribution of susceptibles in July
  suggest that a childhood (under 5y) campaign would not have been
  sufficient to stop the outbreak. These results were consistent whether
  using the age confirmation model or the model based on clinical
  symptoms only. The estimates in June, using the clinical only model
  would have under-estimated RE and resulted in spurious inference that
  an under 5 campaign would have been sufficient to stop the outbreak.
  The estimates in June, using the age confirmation model, resulted in
  wide credible intervals for RE and thus the necessary vaccination
  threshold to stop the outbreak; however, the age confirmation model
  was consistent with the inference using the July data, that a campaign
  targeting children under 5 years of age was unlikely to be sufficient
  to stop the outbreak.
  \end{quote}
\item
  \begin{quote}
  The proposed model allows for formal propagation of uncertainty
  through to evaluation of campaigns
  \end{quote}
\item
  \begin{quote}
  Results suggest that adult targeted campaigns would have been
  necessary to achieve necessary reduction of transmission rate to
  RE\textless{}1.
  \end{quote}
\item
  \begin{quote}
  Local demographic and vaccine history dynamics are not consistent with
  the large number of adult susceptibles observed in the outbreak, which
  suggests role of susceptible immigration in the build-up of outbreak
  risk.
  \end{quote}
\item
  \begin{quote}
  The fitted model for excess susceptibles is consistent with
  independent estimates of migration as a risk factor for measles cases
  (REF: Camargo et al) and age-specific migration rates, which indicate
  that the age classes between 15-25 years had the highest fraction of
  migrants.
  \end{quote}
\item
  \begin{quote}
  In the absence of susceptible migrants, RE would have been
  \textless{}1. Targeted strategies to prevent the build-up of
  susceptibles due to migration could have limited or prevented
  outbreak.
  \end{quote}
\end{enumerate}