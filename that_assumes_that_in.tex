\subsection{\texorpdfstring{\emph{Disease Dynamics
Model}}{Disease Dynamics Model}}\label{disease-dynamics-model}

\emph{Initial Susceptible Population}

We used the population age structure in 1996 and the history of vaccination (through routine immunization and campaigns) prior to 1997 to estimate the initial number of susceptible individuals in each age class at the beginning of the 1997 outbreak.  We modeled the number of susceptible individuals in each annual age class at the beginning of 1997 \(S_{a,0}\) as a finite mixture of local susceptibles \(S_{a,0}^{(L)}\) and excess susceptibles \(S_{a,0}^{(E)}\), where excess susceptibles are defined as those that cannot be predicted from local demographic processes.  We model local susceptibles, \(S_a^{(L)}\) as a binomial draw from the Sao Paulo population:

\begin{equation}
S_a^{(L)} \sim \text{Binomial}(N_a, p_s)
\end{equation}

where \(N_a\) is the population size in age class \(a\) and \(p_s\) the realized susceptibility. Uncertainty in this probability was specified with a beta distribution having expected value \(\mu_s\), which was calculated based on historical immunization activity in the city:

\begin{equation}
\begin{split}
\mu_s &= \pi^{(\text{vacc})} \pi^{(\text{SIA})}\pi^{(\text{nat})} \\
&= (1- 0.85 V_a) \prod_{j=1}^2 \left[1- \psi_j I_{a,j} \right] \left[\exp \sum_{y=0}^{a-1} \theta_{a-y} \right],
\end{split}
\end{equation}

where $\pi^{(\text{vacc})}$, $\pi^{(\text{SIA})}$, and $\pi^{(\text{nat})}$ are proportions of residual susceptibility following routine vaccination, supplemental immunization activities (SIA) and natural immunity, \(V_a\) is the routine vaccination coverage experienced by age class \(a\) when they were eligible for routine vaccination between 9 and 12 months of age. We assume that efficacy of the routine dose is 85\% \cite{Uzicanin_2011}\footnote{We assume efficacy is a function of age, hence 85\% at 9 months (largely due to interference with maternal immunity), with higher efficacy for older age groups.}, while SIAs conducted in 1987 and 1992 were assumed to result in 90\% reduction of susceptibles in their targeted age classes \cite{1959156}. Further, \(\psi_{j}\) is the coverage of the \(j\)\textsuperscript{th} SIA and \(I_{a,j}\) is an indicator function that is 1 if age class a was eligible for the \(j\)\textsuperscript{th} SIA and 0 otherwise, and \(\theta_i\) the force of infection in year i (\emph{i.e.}, the contribution of natural immunity).  The force of infection is assumed to be \(1/A\) in the absence of vaccination, where \(A=\frac{L}{R_0}\) \cite{Anderson_1981} is the mean age of infection in the absence of vaccination, L is the life expectancy, and \(R_0\) is the basic reproduction number. The force of infection \(\theta_i\) in the presence of vaccination is then approximated as \((1-V_a)/A\) \cite{Anderson_1981}.  The basic reproduction number \(R_0\) is estimated below.  We present an analysis of sensitivity to assumptions about historical vaccination coverage in the supplement.

We modeled excess susceptibles using a data augmentation approach. A latent number of susceptibles, constrained to a value uniformly distributed between zero and 1 million, were added to the model. These excess susceptibles were assigned an age distribution, modeled as Gaussian with unknown mean \(\mu_{age}\) and standard deviation \(\sigma_{age}\) and added to the resident susceptibles for a total susceptible estimate in the S\~{a}o Paulo population: \(S_a = S_a^{(L)} + S_a^{(E)}\). Because the number and distribution of excess susceptibles were modeled as latent variables, the estimates of their values were determined by the data, rather than being specified using prior information. This provided a means of accounting for infections that may not have been provided by a local population of susceptibles, but potentially by a pool of non-resident susceptibles migrating from areas of low immunity. We also fit an alternative version of the model assuming no excess susceptibles (\emph{i.e.} the distribution of susceptibles is forced to be consistent with local demographics), to assess the effect of including excess susceptibles explicitly in our model.

We modeled the number of measles cases in each age class \(a\) and time step \(t\) as a Poisson random variable:

\begin{equation}
I_a \sim \text{Poisson}\left(S_{a,t-1} \frac{I_{t-1}B_a}{N} \right),
\end{equation}

where \(S_{a,t-1}\) is the number of susceptibles in age class a at time \(t-1\), \(I_{t-1}\) a row vector of the number of infected individuals in each age class at time \(t-1\), and \(B_a\) is the \(a\)th column of the Who Acquires Infection From Whom (WAIFW) matrix.  We model the WAIFW matrix as an assortative matrix \(B\)

\begin{equation}
B = \left[{
\begin{array}{ccccc}
  {\beta} & {\beta \delta} & \ldots & {\beta \delta^{k-2}} & {\beta \delta^{k-1}}  \\
  {\beta \delta} & {\beta} & \ldots & {\beta \delta^{k-3}} & {\beta \delta^{k-2}} \\
{\beta \delta^2} & {\beta \delta} & \ldots & {\beta \delta^{k-4}} & {\beta \delta^{k-3}}  \\
  \vdots & \vdots & \ddots & \vdots & \vdots \\
  {\beta \delta^{k-1}} & {\beta \delta^{k-2}} & \ldots & {\beta \delta} & {\beta}  \\
\end{array}
}\right]
\end{equation}

that assumes that interaction among age groups declines exponentially with difference in age. The basic reproduction number, R\textsubscript{0}, in the model of the initial population size is taken as the dominant eigenvalue of the matrix B.

Our assumed contact matrix structure is simple, reflecting the fact that there is little information in the available data to inform a more complex parameterization. Nevertheless, the decay parameter allows for some flexibility in transmission dynamics; if there is little difference in the contact rate among age groups, the estimated $\delta$ will be close to one, while lower (higher) contact among disparate groups will result in $\delta$ estimates less (greater) than one. We include results from an alternative structure that assumes no age-related decay in contact rate in the supplement.