For supplemental immunization, rather than assuming a fixed 20\% residual susceptibility (i.e. 20\% of children susceptible to measles prior to the SIA remain suscpetible after the SIA, due to both coverage an efficacy) during the years 1984 and 1991, we sampled the value from a uniform distribution between 0.2 and 0.325, reflecting the possibility that the residual susceptibility could be higher (but not lower) than the number used in the model.

The result of these added priors was to increase the uncertainty in estimates of population susceptibility for the lab confirmation model, but not the clinical confirmation model (Figure A2). With this uncertainty included, the lab confirmation and clinical confirmation models differ in the assessment of whether a campaign targeting children under 5 years would be sufficient to reduce $R_E < 1$ (Figure A2). 